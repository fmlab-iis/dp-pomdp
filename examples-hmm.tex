
\begin{figure}
  \centering
  \resizebox{.8\columnwidth}{!}{
    \begin{tikzpicture}[->,>=stealth',shorten >=1pt,auto,node
      distance=2cm,node/.style={circle,draw}]
      \node[node] (i0) at ( -2,  1) { $0$ };
      \node[node] (i1) at ( -2,  0) { $1$ };
      \node[node] (i2) at ( -2, -1) { $2$ };

      \node[node] (o0) at (  2,  1) { $\underline{0}$ };
      \node[node] (o1) at (  2,  0) { $\underline{1}$ };
      \node[node] (o2) at (  2, -1) { $\underline{2}$ };

      \hide{
      \node at (1.5,  1) {
        $
        \begin{array}{l}
          0(1),1(0),2(0)
        \end{array}
        $ };
      \node at (1.5,  0) {
        $
        \begin{array}{l}
          0(0),1(1),2(0)
        \end{array}
        $ };
      \node at (1.5, -1) {
        $
        \begin{array}{l}
          0(0),1(0),2(1)
        \end{array}
        $ };
      }

      \draw[->,very thick] (i0) -- (o0);   % 2/3
      \draw[->,ultra thin]  (i0) -- (o1);   % 1/6
      \draw[->,ultra thin]  (i0) -- (o2);   % 1/6

      \draw[->,semithick] (i1) -- (o0);               % 1/3
      \draw[->,semithick] (i1) -- (o1);               % 1/3
      \draw[->,semithick] (i1) -- (o2);               % 1/3

      \draw[->,ultra thin]  (i2) -- (o0);   % 1/6
      \draw[->,ultra thin]  (i2) -- (o1);   % 1/6
      \draw[->,very thick] (i2) -- (o2);   % 2/3

\hide{      
      \path
      (i0) edge node[above] { $\frac{2}{3}$ } (o0)
      (i0) edge node { $\frac{1}{6}$ } (o1)
      (i0) edge node[below] { $\frac{1}{6}$ } (o2)

      (i1) edge node[above] { $\frac{1}{3}$ } (o0)
      (i1) edge node { $\frac{1}{3}$ } (o1)
      (i1) edge node[below] { $\frac{1}{3}$ } (o2)

      (i2) edge node[above] { $\frac{1}{6}$ } (o0)
      (i2) edge node { $\frac{1}{6}$ } (o1)
      (i2) edge node[below] { $\frac{2}{3}$ } (o2)
      ;
    }
      \end{tikzpicture}
    }
  \caption{Truncated $\frac{1}{2}$-Geometric Mechanism}
  \label{figure:geometric-mechanism}
\end{figure}

Consider the truncated $\frac{1}{2}$-geometric mechanism shown in
Figure~\ref{figure:geometric-mechanism}. At the state $\underline{0}$,
$0$ is observed with probability $1$ but $1$ and $2$ cannot be
observed. Similarly, the states $\underline{1}$ and $\underline{2}$
only observe $1$ and $2$ respectively. A prior knowledge is an
information state where the probabilities at the states $\underline{0}$,
$\underline{1}$, and $\underline{2}$ are $0$. That is, the probability
distribution on the states $0$, $1$, and $2$ is known \emph{a priori}.
In the figure, thin arrows denote transitions with probability
$\frac{1}{6}$; normal arrows denote transitions with probability
$\frac{1}{3}$; thick arrows denote transitions with probability
$\frac{2}{3}$. For instance, the state $0$ can transit to the state
$\underline{0}$ with probability $\frac{2}{3}$ while it can transit to
the state $\underline{1}$ with probability $\frac{1}{6}$.
