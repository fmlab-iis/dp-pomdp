
\begin{algorithm}
  \begin{algorithmic}[1]
    \Require{$H = ((S, p), \Obs, o)$ : an HMM; 
      $\pi, \tau$ : state distributions on $S$; $c$ : a
      non-negative real number; $k$ : a positive integer}
    \Ensure{An SMT query $q$ such that $q$ is unsatisfiable iff
      $\Pr (\overline{\omega} | \pi, H) \leq c \cdot
       \Pr (\overline{\omega} | \tau, H)$ for every observation
       sequences $\overline{\omega}$ of length $k$}
     \Function{PufferfishCheck}{$H$, $\pi_0$, $\pi_1$, $c$, $k$}
       \For{$s \in S$}
         \State{$\alpha_0(s) \leftarrow
           \textmd{\textsc{Product}} (\pi(s),
           \textmd{\textsc{Select}}(\mathsf{w_0}, \Omega, o(s, \bullet)))$}
         \State{$\beta_0(s) \leftarrow
           \textmd{\textsc{Product}} (\tau(s),
           \textmd{\textsc{Select}}(\mathsf{w_0}, \Omega, o(s, \bullet)))$}
       \EndFor
       \For{$t \leftarrow 1$ \textbf{to} $k - 1$}
         \For{$s' \in S$}
           \State{$\alpha_t(s') \leftarrow
             \textmd{\textsc{Product}} (
             \textmd{\textsc{Dot}} (\alpha_{t-1}, p (\bullet, s')),
             \textmd{\textsc{Select}} (\mathsf{w_t}, \Omega, o(s', \bullet)))
             $}
           \State{$\beta_t(s') \leftarrow
             \textmd{\textsc{Product}} (
             \textmd{\textsc{Dot}} (\beta_{t-1}, p (\bullet, s')),
             \textmd{\textsc{Select}} (\mathsf{w_t}, \Omega, o(s', \bullet)))
             $}
         \EndFor
       \EndFor
       \State{\Return{
           $\bigwedge_{t=0}^{k-1} \mathsf{w_t} \in \Obs \wedge
           \textmd{\textsc{Gt}} (
           \textmd{\textsc{Sum}} (\alpha_{k-1}),
           \textmd{\textsc{Product}} (c,
           \textmd{\textsc{Sum}} (\beta_{k-1}))
           )$
       }}
     \EndFunction
 \end{algorithmic}
 \caption{Pufferfish Check}
 \label{algorithm:pufferfish-check}
\end{algorithm}

Our next goal is to develop an algorithm to check
$\epsilon$-Pufferfish privacy on any given HMM. Since the problem is
NP-hard, we will employ Satisfiability Modulo Theories (SMT) solvers
in our algorithm. For all observation sequences of length $k$, we will 
construct an SMT query to find a sequence violating
$\epsilon$-Pufferfish privacy. If no such sequence can be found, the
given HMM satisfies $\epsilon$-Pufferfish privacy for observation
sequences of length $k$.

Let $H = ((S, p), \Obs, o)$ be an HMM, $\pi, \tau$ two initial state
distributions on $S$, $c \geq 0$ a real number, and $k$ a positive
integer. For a fixed observation sequence $\overline{\obs}$,
computing the probability $\Pr (\overline{\obs} | \pi, 
H)$ of observing $\overline{\obs}$ from $\pi$ can be done in
polynomial time~\cite{R:89:ATHMM}. It is straightforward to check if
$\Pr (\overline{\obs} | \pi, H) > c \cdot \Pr (\overline{\obs} |
\tau, H)$ for any fixed observation sequence $\overline{\obs}$. One
simply computes the respective probabilities and then checks the
inequality. 

Our algorithm exploits the efficient algorithm for computing the
probability of observation sequences. Rather than a fixed observation 
sequence, we declare $k$ SMT variables $\mathsf{w_0}, \mathsf{w_1},
\ldots, \mathsf{w_{k-1}}$ for observations at each step. The
observation at each step is determined by one of the $k$ variables. To
this end, we define the SMT expression
$\textmd{\textsc{Select}} (\mathsf{w}, \{ \obs_1, \obs_2, \ldots,
\obs_m\}, o (s, \bullet))$  equal to $o (s, \obs)$ when the SMT
variable $\mathsf{w}$ is $\obs$. It is easy to formulate by
the SMT $\mathsf{ite}$ (if-then-else) expression:
\begin{align*}
  \mathsf{ite} (\mathsf{w} = \obs_1, o (s, \obs_1),
  \mathsf{ite} (\mathsf{w} = \obs_2, o (s, \obs_2),
  \ldots
  ,\mathsf{ite} (\mathsf{w} = \obs_m, o (s, \obs_m), \mathsf{w})
  \ldots))
\end{align*}

Using $\textmd{\textsc{Select}} (\mathsf{w}, \{ \obs_1, \obs_2,
\ldots, \obs_m\}, o (s, \bullet))$, it is easy to construct an SMT
expression to compute $\Pr (\overline{\mathsf{w}} | \pi, H)$ where
$\overline{\mathsf{w}}$ is a sequence of SMT variables ranging over
the observations $\Obs$ (Algorithm~\ref{algorithm:pufferfish-check}).
Recall the equations (\ref{hmm:basis}) and (\ref{hmm:inductive}). We
simply replace the expression $o (s, \obs)$ with
$\textmd{\textsc{Select}} (\mathsf{w}, \{ \obs_1, \obs_2, \ldots,
\obs_m\}, o (s, \bullet))$ to leave the observation determined by the
SMT variable $\mathsf{w}$. In the algorithm, we also use auxiliary
functions. 
$\textmd{\textsc{Product}}(\mathit{smtExp}_0, \mathit{smtExp}_1,
\ldots, \mathit{smtExp}_m)$
returns the SMT expression denoting the product of
$\mathit{smtExp}_0$, $\mathit{smtExp}_1, \ldots,$
$\mathit{smtExp}_m$. Similarly, 
$\textmd{\textsc{Sum}}(\mathit{smtExp}_0,$ $\mathit{smtExp}_1, \ldots,$
$\mathit{smtExp}_m)$ returns the SMT expression for the sum
of $\mathit{smtExp}_0$, $\mathit{smtExp}_1, \ldots,$
$\mathit{smtExp}_m$. $\textmd{\textsc{Gt}} (\mathit{smtExp}_0,
\mathit{smtExp}_1)$ returns the SMT expression for
$\mathit{smtExp}_0$ greater than $\mathit{smtExp}_1$.
Finally, 
$\textmd{\textsc{Dot}} ([\mathsf{a_0}, \mathsf{a_1}, \ldots,$
$\mathsf{a_n}], [\mathsf{b_0}, \mathsf{b_1}, \ldots, \mathsf{b_n}])$
returns the SMT expression for the inner product of the two lists of
SMT expressions, namely,
\[
  \textmd{\textsc{Sum}} (
  \textmd{\textsc{Product}}(\mathsf{a_0}, \mathsf{b_0}),
  \textmd{\textsc{Product}}(\mathsf{a_1}, \mathsf{b_1}),
  \ldots,
  \textmd{\textsc{Product}}(\mathsf{a_n}, \mathsf{b_n})
  ).
\]

Algorithm~\ref{algorithm:pufferfish-check} is summarized in the
following theorem. 

\begin{theorem}
  Let $H = ((S, p), \Obs, o)$ be an HMM, $\pi, \tau$ state
  distributions on $S$, $c > 0$ a real number, and $k > 0$ an
  integer. Algorithm~\ref{algorithm:pufferfish-check} returns an SMT
  query such that the query is unsatisfiable iff
  $\Pr (\overline{\omega} | \pi, H) \leq c \cdot
  \Pr (\overline{\omega} | \tau, H)$ for every observation
  sequences $\overline{\omega}$ of length $k$.
\end{theorem}


We have implemented our algorithm using \zpython. All examples in
Section~\ref{section:hmm} are checked with our implementation. In
\textit{Example~4}, contracting the disease is independent with
probability $p > 0$. Even though the query contains non-linear
constraints over real numbers, \zpython still proves that the
$\frac{1}{2}$-geometric mechanism satisfies $\ln(2)$-Pufferfish
privacy for any probability $0 < p < 1$. In \textit{Example~5}, it is
assumed that the probability of $m$ individuals contracting the
disease is $p_m > 0$ for $m = 0, 1, 2$. The probabilities $(p_0, p_1,
p_2) = (\frac{1}{2}, \frac{1}{4}, \frac{1}{4})$ are in fact found by
\zpython. The SMT solver is able to find a non-binomial initial state
distribution to satisfy $\ln(2)$-Pufferfish privacy. 