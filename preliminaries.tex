
A \emph{Markov decision process (MDP)} $M = (\Act, S, p, L)$ consists of
a finite set $\Act$ of \emph{actions}, a set $S$ of \emph{states}, a
\emph{labelling function} $L : S \rightarrow 2^{\AP}$, and 
a \emph{transition distribution} $p : \Act \times S \times S
\rightarrow \bbfRU$ such that $\sum_{t \in S} p (s, a, t) = 1$
for every $a \in \Act$ and $s \in S$. If the set of states is finite,
we say $M$ is a \emph{finite} MDP.

A \emph{partially observable Markov decision process (POMDP)} $P =
(M, \Obs, r)$ is a finite MDP $M = (\Act, S, p, L)$ with a finite set
$\Obs$ of \emph{observations} and an \emph{observation distribution} $o
: \Act  \times S \times \Obs \rightarrow \bbfRU$ such that $\sum_{\obs
  \in \Obs} o (a, s, \obs) = 1$ for every $a \in \Act$ and $s \in S$.
A POMDP models an MDP with incomplete information. The current state
of the underlying MDP is not known precisely. Rather, schedules are
made by observations and state distributions. 

Formally, let $\Info (S) = \{ \info \in S \rightarrow \bbfRU : \sum_{s \in
 S} \info (s) = 1 \}$ be the set of probability distributions on $S$. 
Consider an \emph{initial state distribution} $\info_0 \in \Info (S)$. A
\emph{history} $H_i$ up to time $i$ is a sequence $\info_0, a_1, \obs_1,
a_2, \obs_2, \ldots, a_{i-1}, \obs_{i-1}$. At time $i$, $H_i$ is the
only information available externally. If an action $a_i$ is chosen
at time $i$ based on $H_i$, the following steps are carried out
consecutively:
\begin{enumerate}
\item If the current state is $s_{i-1}$, the POMDP moves to the state
  $s_i$ with probability $p (s_{i-1}, a_i, s_i)$;
\item An observation $\obs_i$ is received with probability $o (a_i, s_i,
  \obs_i)$; and
\item The history $H_{i+1}$ becomes $H_i, a_i, \obs_i$.
\end{enumerate}
Note that histories do not contain states but the underlying MDP
updates its state as usual. Based on this formalization, we can now
define schedulers for POMDP. Let $\Xi$ be the set of histories. A
\emph{scheduler} $\sch{S} : \Xi \rightarrow \Act$ for the POMDP $P
= (M, \Obs, w)$ with $M = (\Act, S, p, L)$ is a function
from histories to actions. 

Interestingly, it is not necessary to keep
histories. For schedulers, it suffices to maintain the current state
distribution. Let $\info \in \Info (S)$ be a state distribution. Define
$T (\info, a, \obs) \in \Info (S)$ by
\begin{eqnarray}
  \Pr(\obs | \info, a) & = &
    \sum_{s, t \in S} \info (s) p (s, a, t) o (a, t, \obs)
\label{eqn:state-distribution-observation}
\\                             
  T (\info, a, \obs) (t) & = &
    \frac{\sum_{s \in S} \info (s) p (s, a, t) o (a, t, \obs)}
         {\Pr (\obs | \info, a)}.
\label{eqn:state-distribution-successor}
\end{eqnarray}
That is, $T (\info, a, \obs)$ is the posterior state distribution given
the prior state distribution $\info$, action $a$, and observation
$\obs$. Now recall that the initial state distribution $\info_0$ is
known. Let $\info_{i-1} \in \Info (S)$ be the state distribution at
time $i - 1$. Suppose a scheduler takes the action $a_{i}$ and
observes $\obs_{i}$ at time $i$. Then $\info_i = T (\info_{i-1}, a_i,
\obs_i)$ is the state distribution at time $i$. A
scheduler can then decide the next action $a_{i+1}$ by $\info_i$
accordingly. 

Let $R (s, a)$ be the reward for the state $s$ on the action
$a$, and $\overline{R} (s)$ be the reward for terminating at the state
$s$.\footnote{Observe that $R (s, a)$ is received when the MDP
\emph{leaves} $s$.} For $0 \leq \beta \leq 1$, the scheduler attaining
the maximal expected reward
\[
  E[\beta^T \overline{R} (s_T) + \sum_{i=0}^{T-1} \beta^i R(s_i, a_{i+1})]
\]
can be obtained by iteration. Let $L_{< \infty} (\Info (S))$ be the set of
bounded real-valued functions on $\Info (S)$. Consider
$\Upsilon : \Info (S) \times \Act \times L_{< \infty} (\Info (S)) \rightarrow
\bbfR$ defined by
\[
  \Upsilon (\info, a, f) = \sum_s \info (s) R (s, a) +
  \beta \sum_{\obs \in \Obs}
  \sum_{s, t \in S} \info (s) p (s, a, t) o (a, t, \obs) f (T (\info, a, \obs)).
\]
For $\info \in \Info (S)$, define
\begin{eqnarray}
  \overline{V}_{T} (\info) & = & \sum_s \info (s) \overline{R} (s)
\label{eqn:terminate-probability}
\\
  \overline{V}_{i} (\info) & = &
    \max_{a \in \Act} \{ \Upsilon (\info, a, \overline{V}_{i+1}) \} 
    \hspace{2em} (0 \leq i < T)
\label{eqn:max-probability}
\end{eqnarray}

It is more convenient to use matrices to simplify various
expressions. Without loss of generality, assume $S = \{ 1, 2, \ldots,
n \}$ when $|S| = n$. A state distribution $\info \in \Info (S)$ is
represented by the column vector $(\info (1), \info (2), \ldots, \info
(n))^T$  where $v^T$ is the transpose of the vector $v$.
For $a \in \Act$, let $P^a = [p_{ij}^a]_{|S|
  \times |S|}$ denote the transition matrix where $p_{ij}^a = p (i, a,
j)$. For $a \in \Act, \obs \in \Obs$, let $O^a (\obs) = [
o^a_{ij} ]_{|S| \times |S|}$ be the diagonal matrix where
\begin{eqnarray*}
  o^a_{ij} & = & \left\{
                 \begin{array}{ll}
                   o (a, i, \obs) & \textmd{ if } i = j\\
                   0 & \textmd{ otherwise}
                 \end{array}
                 \right.
\end{eqnarray*}
Let $e$ be the $n$-column vector whose entries are $1$.
Then $(\ref{eqn:state-distribution-observation})$ and
$(\ref{eqn:state-distribution-successor})$ rewrite to
\begin{eqnarray}
  \Pr (\obs | \info, a) & = & \info^T P^a O^a (\obs) e \\
  T (\info, a, \obs) & = & \frac{\info^T P^a O^a (\obs)}
                                {\Pr (\obs | \info, a)}
\end{eqnarray}
Similarly, let $R^a = (R (a, 1), R (a, 2), \ldots, R (a, n))^T$ be the
vector of rewards associated with the actcion
$a$. Equations~$(\ref{eqn:terminate-probability})$ and 
$(\ref{eqn:max-probability})$ rewrite to 
\begin{eqnarray}
  \overline{V}_{T} (\info) & = & \info^T \overline{R}\\
  \overline{V}_i (\info) & = &
        \max_{a \in \Act} \left\{ \info^T [
        R^a + 
        \beta \sum_{\obs \in \Obs}
          P^a O^a (\obs) \overline{V}_{i+1} ] \right\}
          \hspace{2em}(0 \leq i < T)
\end{eqnarray}

\hide{
\begin{eqnarray*}
  \overline{V}_{T} (\info) & = & \info^T \overline{R}\\
  \overline{V}_i (\info) & = &
        \max_{a \in \Act} \left\{ \info^T [
        R^a + 
        \beta \sum_{\obs \in \Obs} P^a O^a (\obs) \gamma^{l (\info, a,
                               \obs, i + 1)} ] \right\}\\
  l (\info, a, \obs, i) & = &
      \max_{\gamma \in \Gamma_{i}}
        \{ \info^T P^a O^a (\obs) \gamma \}
\end{eqnarray*}
}