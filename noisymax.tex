
Noisy Max is a simple yet useful data publishing mechanism in
differential privacy~\cite{DR:14:AFDP,DWWZK:18:DVDP}. Consider $n$
queries of the same range, say, the number of patients for $n$
different diseases in a hospital. We are interested in knowing which
has the maximal value among the $n$ queries. A simple
privacy-respecting way to release the information is to add
independent noises to every query results and then return the index of
the maximal noisy results.

Algorithm~\ref{algorithm:noisy-max} shows an instance of Noisy Max by
applying the $\frac{1}{2}$-geometric mechanism to each query with
the discrete range $\{ 0, 1, 2 \}$. The algorithm first computes
a noisy result $\tilde{v}_i$ for $v_i$ using the geometric
mechanism. It then returns the index $r$ of the maximal
$\tilde{v}_r$ among $\{ \tilde{v}_1, \tilde{v}_2, \ldots, \tilde{v}_n
\}$. 
 
% It is important to return indices. Otherwise, this is not DP.

\begin{algorithm}
  \begin{algorithmic}[1]
    \Require{$0 \leq v_1, v_2, \ldots, v_n \leq 2$}
    \Ensure{The index $r$ with the maximal $v_r$ among $v_1, v_2, \ldots, v_n$}
    \Function{NoisyMax}{$v_1, v_2, \ldots, v_n$}
      \For{each $v_i$}
        \Match{$v_i$}
        \Comment{apply $\frac{1}{2}$-geometric mechanism to $v_i$}
          \lCase{$0$}
                {$\tilde{v}_i \leftarrow 0, 1, 2$ with probability
                 $\frac{2}{3}, \frac{1}{6}, \frac{1}{6}$}
          \lCase{$1$}
                {$\tilde{v}_i \leftarrow 0, 1, 2$ with probability
                 $\frac{1}{3}, \frac{1}{3}, \frac{1}{3}$}
          \lCase{$2$}
                {$\tilde{v}_i \leftarrow 0, 1, 2$ with probability
                 $\frac{1}{6}, \frac{1}{6}, \frac{2}{3}$}
        \EndMatch
      \EndFor
      \State{find $1 \leq r \leq n$ such that
        $\tilde{v}_r \geq \tilde{v}_j$ for every $1 \leq j \leq n$}
      \State{\Return {$r$} }
    \EndFunction
  \end{algorithmic}
  \caption{Noisy Max}
  \label{algorithm:noisy-max}
\end{algorithm}

To check Algorithm~\ref{algorithm:noisy-max}, consider the Markov
chain in Figure~\ref{figure:hmm-noisy-max}. The possible values of
$v_i$ are shown on the top. In the figure, node labels denote the
values of $v_1$, $v_2$, and $v_3$. For instance, the node labeled
$011$ represents the inputs $(v_1, v_2, v_3) = (0, 1, 1)$. Similarly,
the inputs $(v_1, v_2, v_3) = (1, 2, 0)$ and $(v_1, v_2, v_3) = (2, 0,
2)$ are shown.
The noisy values are the labels of the nodes at the bottom. 
Hence the node labeled $0\underline{2}1$ represents the noisy values
$\tilde{v}_1 = 0$, 
$\tilde{v}_2 = 2$, and $\tilde{v}_3 = 1$. The underline denotes a
maximal noisy value. For the node labeled $0\underline{2}1$, the index
$2$ is returned (or observed) because $\tilde{v}_2$ has the maximal
value among $\tilde{v}_1$, $\tilde{v}_2$, and $\tilde{v}_3$.

Arrows again denotes probabilistic transitions. Hence the probability
of moving to $(\tilde{v}_1, \tilde{v}_2, \tilde{v}_3) = (0, 2, 1)$
from $(v_1, v_2, v_3) = (0, 1, 1)$ is $\frac{2}{3} \cdot \frac{1}{3}
\cdot \frac{2}{3}$ by the $\frac{1}{2}$-geometric
mechanism. Similarly, the probability of obtaining the same noisy
values from $(v_1, v_2, v_3) = (1, 2, 0)$ is $\frac{1}{3} \cdot
\frac{2}{3} \cdot \frac{1}{3}$.

\begin{figure}
  \centering
    \resizebox{.8\columnwidth}{!}{
    \begin{tikzpicture}[->,>=stealth',shorten >=1pt,auto,node
      distance=2cm,node/.style={circle,draw}]

      \node at (-2.3, 1) { $\cdots$ };
      \node[node,scale=.67] (v011) at (-1.6, 1) { $011$ };
      \node at ( -.8, 1) { $\cdots$ };
      \node[node,scale=.67] (v120) at (   0, 1) { $120$ };
      \node at ( 1.0, 1) { $\cdots$ };
      \node[node,scale=.67] (v202) at ( 2.0, 1) { $202$ };
      \node at ( 2.8, 1) { $\cdots$ };

      \node at (-1.5, -1) { $\cdots$ };
      \node[node, scale=0.67] (021) at (0, -1) { $0\underline{2}1$ };
      \node at ( 1.5, -1) { $\cdots$ };
    
      \path
      (v011) edge [left,below] node [rotate=310]
      { $\frac{2}{3} \cdot \frac{1}{3} \cdot \frac{2}{3}$ }
      (021)

      (v120) edge [right,above] node [rotate=270]
      { $\frac{1}{3} \cdot \frac{2}{3} \cdot \frac{1}{3}$ }
      (021)

      (v202) edge [right,below] node [rotate=45]
      { $\frac{1}{6} \cdot \frac{1}{6} \cdot \frac{1}{3}$ }
      (021)
      ;
    \end{tikzpicture}
    }
  \caption{Hidden Markov Model for Noisy Max}
  \label{figure:hmm-noisy-max}
\end{figure}



